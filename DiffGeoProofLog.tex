\documentclass[11pt]{article}
%you can look for fun LaTeX packages to use hereasdf

\usepackage{amsmath}
\usepackage{amssymb}
\usepackage{fancyhdr}
\usepackage{amsthm}

\usepackage{graphicx}
\usepackage{dcolumn}
\usepackage{bm}

%fun commands for fun sets
%make sure to use these in math mode
\newcommand{\Z}{\mathbb{Z}}
\newcommand{\R}{\mathbb{R}}
\newcommand{\N}{\mathbb{N}}
\newcommand{\C}{\mathbb{C}}
\newcommand{\m}{\mathcal{M}}
\newcommand{\Tt}{\mathcal{T}}
\newcommand{\pa}{\partial}
\newcommand{\dD}{\mathcal{D}}



\oddsidemargin0cm
\topmargin-2cm    
\textwidth16.5cm   
\textheight23.5cm  

\newcommand{\question}[2] {\vspace{.25in} \hrule\vspace{0.5em}
\noindent{\bf #1: #2} \vspace{0.5em}
\hrule \vspace{.10in}}
\renewcommand{\part}[1] {\vspace{.10in} {\bf (#1)}}

\newcommand{\myname}{Alex Havrilla}
\newcommand{\myandrew}{alumhavr}
\newcommand{\myhwnum}{Hw 2}


\setlength{\parindent}{0pt}
\setlength{\parskip}{5pt plus 1pt}
 
\pagestyle{fancyplain}
\lhead{\fancyplain{}{\textbf{HW\myhwnum}}}      % Note the different brackets!
\rhead{\fancyplain{}{\myname\\ \myandrew}}
\chead{\fancyplain{}{\mycourse}}

\linespread{1.3}

\begin{document}

\newtheorem{prop}{Prop}
\newtheorem{lemma}{Lemma}
\newtheorem{remark}{Remark}
\newtheorem{defi}{Def}
\newtheorem{apps}{Application}
\newtheorem{quest}{Question}
\newtheorem{ans}{Answer}
\newtheorem{interest}{Interesting}
\newtheorem{theme}{Theme}
\newtheorem{theorem}{Theorem}
\newtheorem{example}{Example}


\medskip                        

\thispagestyle{plain}
\begin{center}

{\myname}

\myandrew

\myhwnum

\end{center}

\section{Confusions}

For curve $\gamma$, $\frac{d\gamma}{dt}$ is a Vector Field. Gives the vector corresponding to "direction of curve". At a point p

\begin{align*}
	\frac{d\gamma}{dt}|_p[f] = \frac{d}{dt}(f \circ \gamma) |_{t = 0}
\end{align*}

Why can $T_pM$ be thought of as a copy of $\R^n$?

Meridians and parallels on surface S like longitudes(lines with constant width) and latitudes(geodesics with constant height) on sphere.

Geodesic sphere: Seems to be points of distance r away from point p determined by radiating geodesics of length r from p. Denotes $S_r(p)$. 

?How does differential of compositions work?
I think trivially: $df(dg) = d f\circ g$ since
\begin{align*}
	d f\circ g[v] = v[f\circ g] = dg v[f] = df dg v
\end{align*}
Note here I'm not applying all the way: should really be applying resulting vector to functions $M \to \R$. 

?What is the relationship between differential and Affine Connection/Covariant Derivative?

Defining 3 tensors like $R'$ via 
\begin{align*}
	g(R'(X,Y,W),Z) = \langle X,W\rangle\langle Y , Z \rangle - \langle Y,W\rangle \langle X,Z \rangle
\end{align*}

ie. an implicit definition. Much like how we can define directional derivatives.

A lot of the formulas for curvatures don't seem to be typechecking for me?

\section{Content}

\textbf{Claim:} For $X,Y \in \mathbb{X}(\m)$ for some manifold $\m$, we have
\[
	[X,Y] = L_XY
\]

(Note this means lie derivative produces another vector field)

\begin{proof}

	Let $\Phi_t$ be the flow of X. 

	Recall this is defined as $\Phi: \R \times \m \to \m$ via $\Phi_t(p) = \gamma(t)$ where $\gamma$ solves ODE $\gamma'(t) = X(\gamma(t)), \gamma(0) = p$.(A collection of paths over time flowing along vector field X for some initial condition). (So each vector field produces a flow).

	$\forall g \in \dD$, $X|_p[g] = \frac{d\Phi_t(p)}{dt}|_{t=0}[g] = \frac{d}{dt}_{t=0}g(\phi_t(p))$ which is true by definition of the flow

	Let $\psi_s$ be the flow of Y. For $f \in \dD$ set $H(t,s) = f(\Phi_{-t}(\psi_s(\Phi_t(p))))$(flow forward t along X, then s along Y, then back -t along X). 

	Then $\frac{\partial H}{\partial s}_{(t,0)} = Y|_{\phi_t(p)} [f \circ \phi_{-t}]$ since symbolically this is the same as two lines above(making some substitutions).

	Taking a derivative in t yields $\frac{\partial^2 H}{\partial t \partial s}|_{(0,0)} = \frac{d}{dt}|_{t=0} Y|_{\phi_t(p)}[f \circ \phi_{-t}]$

	But we know $L_XY|_p[f] = \frac{d}{dt}|_{t=0} d\phi_{-t}(Y|_{\phi_t(p)})[f]$.

	Recall the lie derivative is defined as 
	\[
		L_X Y|_p = lim_{t\to 0} \frac{d\phi_{-t} Y|_{\phi_t(p)}-Y|_p}{t} = \frac{d}{dt}|_{t=0}d\phi_{-t}(Y_{\phi_t(p)})
	\]
	where we measure the change in Y at a point p against flows forward along X. Ie. the change in Y against X

	And $\frac{d}{dt}|_{t=0} d\phi_{-t}(Y|_{\phi_t(p)})[f] = \frac{d}{dt}|_{t=0} Y|_{\phi_t(p)}[f \circ \phi_{-t}]$. So lie derivative is cross term of second derivative of H.

	Then define $K(r,s,t)$ to show cross terms of second derivative of H also equal to lie bracket. 

\end{proof}


\begin{prop}\textbf{Levi-Civita}
	Let $(\m,g)$ be a Riemanian manifold. Then $\exists$ a unique affine connection $\nabla$ which is compatible with $g$ and symmetric. Such connection is called Riemann connection.
\end{prop}

\begin{proof}
	Suppose $\nabla$ compatible and symmetric connection. Then

	\begin{align*}
		&Xg(Y,Z) = g(\nabla_X Y, Z) + g(Y,\nabla_X Z)\\
		&Yg(Z,X) = g(\nabla_Y Z,X) + g(Z,\nabla_Y X)\\
		&-Zg(X,Y) = g(\nabla_Z X,Y) + g(X,\nabla_Z Y)
	\end{align*}
	which comes from compatibility, since X acting on function yields s.t. corollary to compatibility.(We may say this pointwise since at any ponit we have an integral curve $\gamma$ for X with $\gamma(0) = p$). 

	Then substituting expression via symmetry of the expression $\nabla_X Z - \nabla_Z X = [X,Z]$ and the like and cancelling terms when adding the above three lines yields

	\begin{align*}
		g(\nabla_X Y,Z) = \frac{1}{2}(Xg(Y,Z) + Y g(Z,X) - Zg(X,Y) + g(X,[Y,Z]) + g(Y,[X,Z]) - g(Z,[X,Y]))
	\end{align*}

	so then we define $\nabla_X Y$ to be the vector s.t. the relation holds(note this is how inner product leads to definition of derivative)
\end{proof}

\section{Curvature}

Recall Curvature $R : X(M) \times X(M) \to X(M)$ is defined as

\begin{align*}
	R(X,Y)Z := \nabla_Y\nabla_X Z - \nabla_X\nabla_Y Z + \nabla_{[X,Y]}Z \\
	(X,Y,Z,W) := g(R(X,Y)Z,W)
\end{align*}

Curvature tensor example of covariant tensor. Differential forms example of contravariant tensor(measuring the dual?)

\section{Jacobi Fields}

Jacobi field is vector field along geodisic $\gamma$ which corresponds to a perturbation of $\gamma$ iin the space of geodesic.

Also thought of as tangent vectors in space of geodesics.

We study the following:

\begin{align*}
	(d exp_p)_{tv}(tw) = \frac{df}{ds}(t,0) 
\end{align*}

along $\gamma(t) = exp_p(tv)$

\section{Killing Fields}

\begin{rem}
	Killing fields flows preserve orientation of vectors at points. Flow smoothly(diffeomorphism assumption).
\end{rem}

\section{Jacobi Fields}

\begin{remark}
	A jacobi field is a vector field along a geodesic $\gamma$ which corresponds to a perturbation of $\gamma$ in the space of geodesics.

	Also can be thought of as tanget vectors in space of geodesics(along a geodesic).
\end{remark}

\begin{theme}
	Formalizes connection between curvature and geodesics
\end{theme}

\begin{defi}
	$(d exp_p)_{tv}(tw) = \frac{df}{ds}(t,0)$ along $\gamma(t) = exp_p(tv)$ where $f(t,s) = exp_p t v(s)$
\end{defi}

\begin{remark}
	$(d exp_p)_{tv}$ map from tangent space of tangent space to tangent space of tangent space 
\end{remark}

\begin{defi}
	$0 = \frac{D^2 J}{dt^2} + R(\gamma'(t),J(t))\gamma'(t)$
\end{defi}

\begin{remark}
	Curvatue indicates spread of geodesics
\end{remark}

\begin{prop}
	Jacobi field is determined by initial conditions $J(0),\frac{DJ}{dt}(0)$. 
\end{prop}

\begin{proof}
	Let $e_1(t),...,e_n(t)$ parallel orthonormal fields along $\gamma$. Write
	\begin{align*}
		J(t) = \sum_i f_i(t) e_i(t)
	\end{align*}
	with $a_{ij} = g(R(\gamma'(t),e_i(t))\gamma'(t),e_j(t))$

	Then $\frac{D^2 J}{dt^2} = \sum_i f_i''(t)e_i(t)$ and $R(\gamma',J)\gamma' = \sum_j g(R(\gamma',J)\gamma',e_j)e_j = \sum_{i,j} f_i a_{i,j}e_j$

	since the $e_i$ parallel.

	Jacobi eqution equivalent to $f_j''(t) + \sum_i a_{ij}(t)f_i(t) = 0$ which is a linear second order system.

	Thsu given initial conditions we have $2n$ linearly separable independent jacobi fields along $\gamma$. 
\end{proof}

\begin{theme}
	A lot of these things connected to DEs on the manifold(I guess point of differentials).
\end{theme}

\begin{prop}
	Let M have constant sectional curvature K. $\gamma : [0,l] \to M$ be a geodesic arclength parameterized. Let J jacobi field normal to $\gamma'$. 

	Then $R(\gamma',J)\gamma' = KJ$ ie. curvature of $\gamma',J$ along $\gamma'$ is just $KJ$ ie. multiple of J. J describes turning away of $\gamma$. In some sense an eigenvalue of the curvature. 
\end{prop}

\begin{proof}
	For all $T$ along $\gamma$ we have
	\begin{align*}
		g(R(\gamma',J)\gamma',T) = K(g(\gamma',\gamma')g(J,T) - g(\gamma',T)g(J,\gamma'))
	\end{align*}

	why??? - I think becaue we have 0 sectional curvature so we have this form for the curvature.

	This equals $Kg(J,T)$ as terms cancel via unit norm and parallelism.
\end{proof}

\begin{example}
	Set $M = \R^2$ with $\gamma(t) = (t,0)$. $w(t) = (0,1)$. Then $J(t) = (0,t)$. 

	Showing existence of jacobi field satisfying general form(with 0 sectional curvature).
\end{example}

\begin{theme}
	Typically focus on jacobi fields normal to curve. 
\end{theme}

\begin{prop}
	Can taylor expand $|J(t)|^2$. Involves curvature

	$|J(t)|^2 = t^2 - (1/3)g(R(v,w)v,w)t^4 + ...$

	Notice involves sectional curvature $K(v,w) = g(R(v,w)v,w)$
\end{prop}

\begin{proof}
	Know $J(0) = 0$ with $J'(0) = w$. So 
	\begin{align*}
		&g(J,J)(0) = 0\\
		&(g(J,J))'(0) = 2g(J,J')(0) = 0 \text{ via a product rule likely}\\
		& (g(J,J))''(0) = 2g(J',J')(0) + 2 g(J'',J)(0) = 2 \text{ since second term cancels(parallelism?)}
	\end{align*}

	Last expression also probably comes from DE

	Via the DE also get third order term 0 and fourth order term $8g(J',J''')(0)$

	We have
	\begin{align*}
		J'''(0) = -\frac{D}{dt}|_{t =0}(R(\gamma'(t),J(t))\gamma'(t))
	\end{align*}

	via DE. Substituting this into expression gives us what we want.


\end{proof}

\begin{remark}
	\textbf{Geometric Interpretation:}

	Geodesics $t \to exp_p(tv(s))$ spread from $t \to exp_p(tv)$ at a rate which differs from that of flat space by $\approx-1/6 K(p,\sigma)t^3$.
\end{remark}

\subsection{Conjugate Points}

\begin{remark}
	Conjugate point sinks for tangent geodesics along $\gamma$
\end{remark}

\begin{example}
	Consdier $\mathbb{S}^2$. Antipodes conjugate along $(cos(t),sin(t),0)$ for $t \in [0,\pi]$ with $J(t) = (0,0,sin(t))$
\end{example}

\begin{example}
	Cut locus of $p \in S^n$ is $\{-p\}$
\end{example}

\begin{prop}
	A point $q = \gamma(t_0)$ is conjugate to p along $\gamma \iff$ $v_0 : = t_0 \gamma'(0)$ is a critical point of $exp_p$. also $mult = dim(ker((d exp_p)_{v_0}))$
\end{prop}

\begin{proof}
	We say $q$ conjugate $\iff$ we can find a J s.t. $J(0) = J(t_0) = 0$. We set $v = \gamma'(0)$ and $w = J'(0)$. 

	We know $J(t) = (dexp_p)_{tv}(tw)$. So $0 = ((dexp_p)_0)(0) = (dexp_p)_{t_0v}(t_0 w)$ 
\end{proof}

\begin{prop}
	$g(J,\gamma') = g(J'(0),\gamma'(0))t + g(J(0),\gamma'(0))$
\end{prop}

\begin{proof}
	Compute
	
		$g(J',\gamma')' = g(J'',\gamma')$ \text{Because \gamma'' parallel J'?}
		$= -g(R(\gamma',J)\gamma',\gamma')= 0$ \text{ because we know curvature is constant multiple of gamma'}
	
\end{proof}

\section{Complete Manifolds}

\begin{remark}
	Covering map covers each open set with image of disjoint open sets on which f is a diffeomorphism(f locally a diffeomorphism on open sets)
\end{remark}

\begin{remark}
	The manifold is extendible if it can be isometrically emebedded in larger space(open).
\end{remark}

\begin{example}
	Torus geodesically complete, $\mathbb{R}^n$ geodesically complete, $B(0,1)$ not complete since geodesic runs out of manifold.
\end{example}

\begin{prop}
	If M is extendible then it is not geodesically complete. 
\end{prop}

\begin{proof}
	Suppose M extendible. Then $M \subseteq M'$. Since $M' \backslash M$ is not open(since M' connected) we can find $p \in \partial M = \overline{M}\backslash M$. Let U be normal neighborhood of p. Let $q \in U \cap M$. Then we can find geodesic $\gamma$ from p to q. But this is not in $M$.

	Ie. just look at boundary and find normal neighborhood. 
\end{proof}

\begin{theorem}
	\textit{Hopf-Rinow}
\end{theorem}

\begin{proof}
	Need to show $i) \implies *$ and $i) \iff ii) \iff iii) \iff iv)$
\end{proof}

\section{Isometric Immersions}

\begin{prop}
	For $f : M \to \overline{M}$ isometric immersion, we know $T_{f(p)} f(M) \subseteq T_{f(p)} \overline{M}$ and then $V = V^T + V^N$ where $V^T \in T_{f(p)}f(M)$ and $V^N$ orthogonal ie. $\overline{g}(V^T,V^N) = 0$.
\end{prop}

\begin{remark}
	Can decompose tangent space of $T_p\overline{M}$ as direct sum with tangent space of sub-isometric immersion
\end{remark}

\begin{defi}
	$B(X,Y) = \overline{\nabla}_{\overline{X}} \overline{Y} - \nabla_X Y$ which is in $(T_p M)^{\perp}$ where $\overline{X},\overline{Y}$ extensions
\end{defi}

\begin{remark}
	We measure change in $\overline{Y}$ against change in Y.
\end{remark}





\section{Confusions}

\begin{question}
	Don't have a very good understanding of how to convert bewteen derivatives? Example: look at hw 7.3 computations
\end{question}

\end{document}

