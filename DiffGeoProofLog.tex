\documentclass[11pt]{article}
%you can look for fun LaTeX packages to use hereasdf

\usepackage{amsmath}
\usepackage{amssymb}
\usepackage{fancyhdr}
\usepackage{amsthm}

\usepackage{graphicx}
\usepackage{dcolumn}
\usepackage{bm}

%fun commands for fun sets
%make sure to use these in math mode
\newcommand{\Z}{\mathbb{Z}}
\newcommand{\R}{\mathbb{R}}
\newcommand{\N}{\mathbb{N}}
\newcommand{\C}{\mathbb{C}}
\newcommand{\m}{\mathcal{M}}
\newcommand{\Tt}{\mathcal{T}}
\newcommand{\pa}{\partial}
\newcommand{\dD}{\mathcal{D}}



\oddsidemargin0cm
\topmargin-2cm    
\textwidth16.5cm   
\textheight23.5cm  

\newcommand{\question}[2] {\vspace{.25in} \hrule\vspace{0.5em}
\noindent{\bf #1: #2} \vspace{0.5em}
\hrule \vspace{.10in}}
\renewcommand{\part}[1] {\vspace{.10in} {\bf (#1)}}

\newcommand{\myname}{Alex Havrilla}
\newcommand{\myandrew}{alumhavr}
\newcommand{\myhwnum}{Hw 2}

\newtheorem{prop}{Prop}

\setlength{\parindent}{0pt}
\setlength{\parskip}{5pt plus 1pt}
 
\pagestyle{fancyplain}
\lhead{\fancyplain{}{\textbf{HW\myhwnum}}}      % Note the different brackets!
\rhead{\fancyplain{}{\myname\\ \myandrew}}
\chead{\fancyplain{}{\mycourse}}

\linespread{1.3}

\begin{document}


\medskip                        

\thispagestyle{plain}
\begin{center}

{\myname}

\myandrew

\myhwnum

\end{center}

\section{Contexts}

Geodesicts can be thought of as having constant vector fields?
	Vector field vector does not change along curve.
	Directino does not change. Speed does not change.

To show $g : M^d \to N^d$ local diffeomorphism at p, STS $dg_p$ is full rank.

\section{Confusions}

For curve $\gamma$, $\frac{d\gamma}{dt}$ is a Vector Field. Gives the vector corresponding to "direction of curve". At a point p

\begin{align*}
	\frac{d\gamma}{dt}|_p[f] = \frac{d}{dt}(f \circ \gamma) |_{t = 0}
\end{align*}

For $f : M \to \R$ we have $\frac{d f}{dt}$ is a fucntion from $M \to \R$ where we evaluate:

\begin{align*}
	\frac{df}{dt}|_p = X|_p[f]
\end{align*}

where $X = \frac{d \gamma}{dt}$

$\gamma(t,q,v)$ for geodesic $\gamma$ is transport of $v \in T_qM$ along $\gamma$ by time t? Note that $\gamma$ is unique geodesic determined by $v$ in given direction. Ie. $\gamma_v(0) = p$. Only locally defined(relying on ODE theory).

*u,v seem to be standard coordinates for surface, s parameterization

$\frac{ds}{du}$ is VF on s tangent to curves for fixed $v_0$. 

\section{Proofs}

\textbf{Claim:} For $X,Y \in \mathbb{X}(\m)$ for some manifold $\m$, we have
\[
	[X,Y] = L_XY
\]

(Note this means lie derivative produces another vector field)

\begin{proof}

	Let $\Phi_t$ be the flow of X. 

	Recall this is defined as $\Phi: \R \times \m \to \m$ via $\Phi_t(p) = \gamma(t)$ where $\gamma$ solves ODE $\gamma'(t) = X(\gamma(t)), \gamma(0) = p$.(A collection of paths over time flowing along vector field X for some initial condition). (So each vector field produces a flow).

	$\forall g \in \dD$, $X|_p[g] = \frac{d\Phi_t(p)}{dt}|_{t=0}[g] = \frac{d}{dt}_{t=0}g(\phi_t(p))$ which is true by definition of the flow

	Let $\psi_s$ be the flow of Y. For $f \in \dD$ set $H(t,s) = f(\Phi_{-t}(\psi_s(\Phi_t(p))))$(flow forward t along X, then s along Y, then back -t along X). 

	Then $\frac{\partial H}{\partial s}_{(t,0)} = Y|_{\phi_t(p)} [f \circ \phi_{-t}]$ since symbolically this is the same as two lines above(making some substitutions).

	Taking a derivative in t yields $\frac{\partial^2 H}{\partial t \partial s}|_{(0,0)} = \frac{d}{dt}|_{t=0} Y|_{\phi_t(p)}[f \circ \phi_{-t}]$

	But we know $L_XY|_p[f] = \frac{d}{dt}|_{t=0} d\phi_{-t}(Y|_{\phi_t(p)})[f]$.

	Recall the lie derivative is defined as 
	\[
		L_X Y|_p = lim_{t\to 0} \frac{d\phi_{-t} Y|_{\phi_t(p)}-Y|_p}{t} = \frac{d}{dt}|_{t=0}d\phi_{-t}(Y_{\phi_t(p)})
	\]
	where we measure the change in Y at a point p against flows forward along X. Ie. the change in Y against X

	And $\frac{d}{dt}|_{t=0} d\phi_{-t}(Y|_{\phi_t(p)})[f] = \frac{d}{dt}|_{t=0} Y|_{\phi_t(p)}[f \circ \phi_{-t}]$. So lie derivative is cross term of second derivative of H.

	Then define $K(r,s,t)$ to show cross terms of second derivative of H also equal to lie bracket. 

\end{proof}


\begin{prop}\textbf{Levi-Civita}
	Let $(\m,g)$ be a Riemanian manifold. Then $\exists$ a unique affine connection $\nabla$ which is compatible with $g$ and symmetric. Such connection is called Riemann connection.
\end{prop}

\begin{prop}
	Symmetry lemma
	\begin{align*}
		\frac{D }{\del v} \frac{\del s}{\del u} = \frac{D}{\del u} \frac{\del s}{\del v}
	\end{align*}
\end{prop}

\begin{proof}
	
\end{proof}

\end{document}

