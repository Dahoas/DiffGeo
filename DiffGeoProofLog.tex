\documentclass[11pt]{article}
%you can look for fun LaTeX packages to use hereasdf

\usepackage{amsmath}
\usepackage{amssymb}
\usepackage{fancyhdr}
\usepackage{amsthm}

\usepackage{graphicx}
\usepackage{dcolumn}
\usepackage{bm}

%fun commands for fun sets
%make sure to use these in math mode
\newcommand{\Z}{\mathbb{Z}}
\newcommand{\R}{\mathbb{R}}
\newcommand{\N}{\mathbb{N}}
\newcommand{\C}{\mathbb{C}}
\newcommand{\m}{\mathcal{M}}
\newcommand{\Tt}{\mathcal{T}}
\newcommand{\pa}{\partial}
\newcommand{\dD}{\mathcal{D}}



\oddsidemargin0cm
\topmargin-2cm    
\textwidth16.5cm   
\textheight23.5cm  

\newcommand{\question}[2] {\vspace{.25in} \hrule\vspace{0.5em}
\noindent{\bf #1: #2} \vspace{0.5em}
\hrule \vspace{.10in}}
\renewcommand{\part}[1] {\vspace{.10in} {\bf (#1)}}

\newcommand{\myname}{Alex Havrilla}
\newcommand{\myandrew}{alumhavr}
\newcommand{\myhwnum}{Hw 2}

\newtheorem{prop}{Prop}
\newtheorem{defn}{Definition}
\newtheorem{rem}{Remark}

\setlength{\parindent}{0pt}
\setlength{\parskip}{5pt plus 1pt}
 
\pagestyle{fancyplain}
\lhead{\fancyplain{}{\textbf{HW\myhwnum}}}      % Note the different brackets!
\rhead{\fancyplain{}{\myname\\ \myandrew}}
\chead{\fancyplain{}{\mycourse}}

\linespread{1.3}

\begin{document}


\medskip                        

\thispagestyle{plain}
\begin{center}

{\myname}

\myandrew

\myhwnum

\end{center}

\section{Confusions}

For curve $\gamma$, $\frac{d\gamma}{dt}$ is a Vector Field. Gives the vector corresponding to "direction of curve". At a point p

\begin{align*}
	\frac{d\gamma}{dt}|_p[f] = \frac{d}{dt}(f \circ \gamma) |_{t = 0}
\end{align*}

Why can $T_pM$ be thought of as a copy of $\R^n$?

Meridians and parallels on surface S like longitudes(lines with constant width) and latitudes(geodesics with constant height) on sphere.

Geodesic sphere: Seems to be points of distance r away from point p determined by radiating geodesics of length r from p. Denotes $S_r(p)$. 

?How does differential of compositions work?
I think trivially: $df(dg) = d f\circ g$ since
\begin{align*}
	d f\circ g[v] = v[f\circ g] = dg v[f] = df dg v
\end{align*}
Note here I'm not applying all the way: should really be applying resulting vector to functions $M \to \R$. 

?What is the relationship between differential and Affine Connection/Covariant Derivative?

Defining 3 tensors like $R'$ via 
\begin{align*}
	g(R'(X,Y,W),Z) = \langle X,W\rangle\langle Y , Z \rangle - \langle Y,W\rangle \langle X,Z \rangle
\end{align*}

ie. an implicit definition. Much like how we can define directional derivatives.

A lot of the formulas for curvatures don't seem to be typechecking for me?

\section{Content}

\textbf{Claim:} For $X,Y \in \mathbb{X}(\m)$ for some manifold $\m$, we have
\[
	[X,Y] = L_XY
\]

(Note this means lie derivative produces another vector field)

\begin{proof}

	Let $\Phi_t$ be the flow of X. 

	Recall this is defined as $\Phi: \R \times \m \to \m$ via $\Phi_t(p) = \gamma(t)$ where $\gamma$ solves ODE $\gamma'(t) = X(\gamma(t)), \gamma(0) = p$.(A collection of paths over time flowing along vector field X for some initial condition). (So each vector field produces a flow).

	$\forall g \in \dD$, $X|_p[g] = \frac{d\Phi_t(p)}{dt}|_{t=0}[g] = \frac{d}{dt}_{t=0}g(\phi_t(p))$ which is true by definition of the flow

	Let $\psi_s$ be the flow of Y. For $f \in \dD$ set $H(t,s) = f(\Phi_{-t}(\psi_s(\Phi_t(p))))$(flow forward t along X, then s along Y, then back -t along X). 

	Then $\frac{\partial H}{\partial s}_{(t,0)} = Y|_{\phi_t(p)} [f \circ \phi_{-t}]$ since symbolically this is the same as two lines above(making some substitutions).

	Taking a derivative in t yields $\frac{\partial^2 H}{\partial t \partial s}|_{(0,0)} = \frac{d}{dt}|_{t=0} Y|_{\phi_t(p)}[f \circ \phi_{-t}]$

	But we know $L_XY|_p[f] = \frac{d}{dt}|_{t=0} d\phi_{-t}(Y|_{\phi_t(p)})[f]$.

	Recall the lie derivative is defined as 
	\[
		L_X Y|_p = lim_{t\to 0} \frac{d\phi_{-t} Y|_{\phi_t(p)}-Y|_p}{t} = \frac{d}{dt}|_{t=0}d\phi_{-t}(Y_{\phi_t(p)})
	\]
	where we measure the change in Y at a point p against flows forward along X. Ie. the change in Y against X

	And $\frac{d}{dt}|_{t=0} d\phi_{-t}(Y|_{\phi_t(p)})[f] = \frac{d}{dt}|_{t=0} Y|_{\phi_t(p)}[f \circ \phi_{-t}]$. So lie derivative is cross term of second derivative of H.

	Then define $K(r,s,t)$ to show cross terms of second derivative of H also equal to lie bracket. 

\end{proof}


\begin{prop}\textbf{Levi-Civita}
	Let $(\m,g)$ be a Riemanian manifold. Then $\exists$ a unique affine connection $\nabla$ which is compatible with $g$ and symmetric. Such connection is called Riemann connection.
\end{prop}

\begin{proof}
	Suppose $\nabla$ compatible and symmetric connection. Then

	\begin{align*}
		&Xg(Y,Z) = g(\nabla_X Y, Z) + g(Y,\nabla_X Z)\\
		&Yg(Z,X) = g(\nabla_Y Z,X) + g(Z,\nabla_Y X)\\
		&-Zg(X,Y) = g(\nabla_Z X,Y) + g(X,\nabla_Z Y)
	\end{align*}
	which comes from compatibility, since X acting on function yields s.t. corollary to compatibility.(We may say this pointwise since at any ponit we have an integral curve $\gamma$ for X with $\gamma(0) = p$). 

	Then substituting expression via symmetry of the expression $\nabla_X Z - \nabla_Z X = [X,Z]$ and the like and cancelling terms when adding the above three lines yields

	\begin{align*}
		g(\nabla_X Y,Z) = \frac{1}{2}(Xg(Y,Z) + Y g(Z,X) - Zg(X,Y) + g(X,[Y,Z]) + g(Y,[X,Z]) - g(Z,[X,Y]))
	\end{align*}

	so then we define $\nabla_X Y$ to be the vector s.t. the relation holds(note this is how inner product leads to definition of derivative)
\end{proof}

\section{Curvature}

Recall Curvature $R : X(M) \times X(M) \to X(M)$ is defined as

\begin{align*}
	R(X,Y)Z := \nabla_Y\nabla_X Z - \nabla_X\nabla_Y Z + \nabla_{[X,Y]}Z \\
	(X,Y,Z,W) := g(R(X,Y)Z,W)
\end{align*}

Curvature tensor example of covariant tensor. Differential forms example of contravariant tensor(measuring the dual?)

\section{Jacobi Fields}

Jacobi field is vector field along geodisic $\gamma$ which corresponds to a perturbation of $\gamma$ iin the space of geodesic.

Also thought of as tangent vectors in space of geodesics.

We study the following:

\begin{align*}
	(d exp_p)_{tv}(tw) = \frac{df}{ds}(t,0) 
\end{align*}

along $\gamma(t) = exp_p(tv)$

\section{Killing Fields}

\begin{rem}
	Killing fields flows preserve orientation of vectors at points. Flow smoothly(diffeomorphism assumption).
\end{rem}



\end{document}

