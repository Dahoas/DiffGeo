\documentclass[11pt]{article}
%you can look for fun LaTeX packages to use hereasdf

\usepackage{amsmath}
\usepackage{amssymb}
\usepackage{fancyhdr}
\usepackage{amsthm}

\usepackage{graphicx}
\usepackage{dcolumn}
\usepackage{bm}

%fun commands for fun sets
%make sure to use these in math mode
\newcommand{\Z}{\mathbb{Z}}
\newcommand{\R}{\mathbb{R}}
\newcommand{\N}{\mathbb{N}}
\newcommand{\C}{\mathbb{C}}
\newcommand{\m}{\mathcal{M}}
\newcommand{\Tt}{\mathcal{T}}
\newcommand{\pa}{\partial}
\newcommand{\Dd}{\mathcal{D}}
\newcommand{\del}{\partial}



\oddsidemargin0cm
\topmargin-2cm    
\textwidth16.5cm   
\textheight23.5cm  

\newcommand{\question}[2] {\vspace{.25in} \hrule\vspace{0.5em}
\noindent{\bf #1: #2} \vspace{0.5em}
\hrule \vspace{.10in}}
\renewcommand{\part}[1] {\vspace{.10in} {\bf (#1)}}

\newcommand{\myname}{Alex Havrilla}
\newcommand{\myandrew}{alumhavr}
\newcommand{\myhwnum}{Hw 5}

\newtheorem{prop}{Prop}

\setlength{\parindent}{0pt}
\setlength{\parskip}{5pt plus 1pt}
 
\pagestyle{fancyplain}
\lhead{\fancyplain{}{\textbf{HW\myhwnum}}}      % Note the different brackets!
\rhead{\fancyplain{}{\myname\\ \myandrew}}
\chead{\fancyplain{}{\mycourse}}

\linespread{1.3}

\begin{document}

\medskip                        

\thispagestyle{plain}
\begin{center}

{\myname}

\myandrew

\myhwnum

\end{center}

\question{Question 1}

Let $\m$ be a Riemann manifold with affien connection $\nabla$. Let $\gamma : I \to \m$ be a curve. Let $P_{\gamma,t_0,t} : T_{\gamma(t_0)} \m \to T_{\gamma(t)} \m$ be the mapping taking tangent vector $V_0$ at $\gamma(t_0)$ to $V(t)$ where V is the parallel transport of $V_0$ along $\gamma$. 

Let X and Y be vector fields on $\m$. Consider curve $\gamma$ as an integral curve for X. Then $\frac{d \gamma}{d t} = X|_{\gamma(t)}$.

\begin{prop}
Where $\nabla$ is the Riemann connection then
	\begin{align*}
		\nabla_X Y |_{\gamma(t_0)} = \frac{d}{dt}(P^{-1}_{\gamma,t_0,t}Y|_{\gamma(t)}) | _{t = t_0}
	\end{align*}
\end{prop}

\begin{proof}
	Set $p_t = \gamma(t)$ and $p_0 = \gamma(t_0)$. Then at $T_{p_0}\m$ pick an orthonormal basis $v_1,...,v_n$. We extend these to vector fields $V_1$,...,$V_n$ along $\gamma$ using parallel transport. Notice via compatibility of with the metric these stay orthogonal. We can then write $Y = a^iV_i$ and thus compute
	\begin{align*}
		\nabla_X Y|_{\gamma(t_0)} = \nabla_X a^iV_i|_{p_0} = X[a^i]V_i|_{p_0} + a^i \nabla_X V_i |_{p_0}= X[a^i]v_i
	\end{align*}

	where the second term $a^i \nabla_X V_i = 0$ since the $V_i$ are parallel along $\gamma$.

	%Point is to take control of the parallel transport 

	Then in the other direction we compute
	
	\begin{align*}
		\frac{d}{dt}(P^{-1}_{\gamma,t_0,t}Y|_{p_t})|_{t=t_0} &= \frac{d}{dt}(P^{-1}_{\gamma,t_0,t}a^iV_i|_{p_t})|_{t=t_0} = \frac{d}{dt}(a^i|_{p_t}P^{-1}_{\gamma,t_0,t}V_i|_{p_t})|_{t=t_0} \\
		& \frac{d}{dt}(a^i|_{p_t}v_i)|_{t=t_0} =  X[a^i]v_i
	\end{align*}

	where we note parallel transport is linear and thus the inverse is linear. Further $\frac{d}{dt}(a_i v_i) = X[a^i]v_i$ since $X = \frac{d \gamma}{dt}$
	%?why is this?
\end{proof}

\question{Question 2}

Let $\m$, $\overline{\m}$ be as defined with $ : M \to \overline{\m}$ an immersion. Let $g = f^* \overline{g}$ and $\nabla_X Y|_p$ as defined for vector fields $X,Y$ on $\m$.

\begin{prop}
	$\nabla$ as defined is the Riemannain connection on $(\m,g)$.
\end{prop}

\begin{proof}
	To show $\nabla$ is the Riemannian connection on $\m$ it suffices to show it is a connection, symmetric, and compatible with $g$. Then via uniqueness we are done.

	First we show it's a connection. Note both $(df)^{-1}$ and $\Pi_T$ are(locally) linear. Compute

	%?Might have to be careful about how I extend?

	\begin{align*}
		\nabla_{X + Y} Z &= (df)^{-1}\Pi_T(\overline{\nabla}_{\overline{X} + \overline{Y}}df(Z)) = (df)^{-1}\Pi_T(\overline{\nabla}_{\overline{X}} df(Z) + \overline{\nabla}_{\overline{Y}}df(Z)) =\\ &=(df)^{-1}\Pi_T \overline{\nabla}_{\overline{X}} df(Z) + (df)^{-1}\Pi_T \overline{\nabla}_{\overline{Y}} df(Z) = \nabla_X Z + \nabla_Y Z
	\end{align*}

	$\nabla_X(Y_1 + Y_2) = \nabla_X Y_1 + \nabla_X Y_2$ follows similarly. 

	\begin{align*}
		\nabla_X (hY) &= (df)^{-1}\Pi_T(\overline{\nabla}_{df(X)}hdf(Y)) = (df)^{-1}\Pi_T(df(X)[h]df(Y) + h\overline{\nabla}_{df(X)}df(Y))\\
		&= X[h]Y + h \nabla_X Y
	\end{align*}

	Next we show symmetry:

	\begin{align*}
		\nabla_X Y - \nabla_Y X &= (df)^{-1}\Pi_T(\overline{\nabla}_{df(X)}df(Y)) - (df)^{-1}\Pi_T(\overline{\nabla}_{df(Y)}df(X)) \\
		&= (df)^{-1}\Pi_T(\overline{\nabla}_{df(X)}df(Y) - (\overline{\nabla}_{df(Y)}df(X))) = df^{-1}\Pi_T([df(X),df(Y)]) = [X,Y]
	\end{align*}

	%Do I need to be more careful here?

	And finally that it is compatible:

	Recall that the metric is compatible with $\nabla$ if $X[g(Y,Z)] = g(\nabla_X Y,Z) + g(Y,\nabla_X Z)$. We show

	\begin{align*}
		X(Y,Z)_g &= (df)(X)((df)(Y),(df)(Z))_{\overline{g}} \\
		&= \overline{\nabla}_{(df)(X)} ((df)(Y),(df)(Z))_{\overline{g}} = (\overline{\nabla}_{(df)(X)}(df)(Y),df(Z))_{\overline{g}} + (df(Y),\overline{\nabla}_{(df)(X)}(df)(Z))_{\overline{g}} \\
		&= (\nabla_X Y, Z)_g + (Y,\nabla_X Z)_g
	\end{align*}

	Thus by uniqueness we have shown $\nabla_X Y$ is the affine connection $\m$.
\end{proof}

\question{Question 3}

Set $\R_+^2 = \{(x,y) \in \R^2, y > 0\}$

with metric coefficients $g_{11}=g_{22} = \frac{1}{y^2}$ and $g_{12} = 0$. 

\begin{prop}
	The christoffel symbols of the Riemannian connection are $\Gamma_{11}^1 = \Gamma_{12}^2 = \Gamma_{22}^1=0$ and $\Gamma_{11}^2 = \frac{1}{y}, \Gamma_{12}^1= \Gamma_{22}^2 = \frac{-1}{y}$
\end{prop}

\begin{proof}
	It suffices to compute the christoffel symbols.

	First note we have for $y > 0$ ie. the half plance:
	\[G^{-1} = 
			\begin{bmatrix}
				y^2 & 0\\
				0 & y^2\\
			\end{bmatrix}
	\]

	Recall we know for arbitrary symbol
	\begin{align*}
		\Gamma_{ij}^k g^{kl}/2(\frac{\partial g_{il}}{\partial x_j} + \frac{\partial g_{jl}}{\partial x_i} - \frac{\partial g_{ij}}{\partial x_l})
	\end{align*}

	Compute

	\begin{align*}
		\Gamma_{11}^1 = \frac{g^{11}}{2}(\frac{\partial g_{11}}{\partial x} + \frac{\partial g_{11}}{\partial x} - \frac{\partial g_{11}}{\partial x}) + \frac{g^{12}}{2}(\frac{\partial g_{12}}{\partial x} + \frac{\partial g_{12}}{\partial x} - \frac{\partial g_{11}}{\partial x}) = 0
	\end{align*}

	since $\frac{\partial \cdot}{\partial x} = 0 $ and $g^{12} = 0$. 

	A similar computation holds for $\Gamma_{12}^2, \Gamma_{22}^1$
	%?Should I work this out?

	Now compute

	\begin{align*}
		\Gamma_{12}^1 = \frac{g^{11}}{2}(\frac{\partial g_{11}}{\partial y} + \frac{\partial g_{2}}{\partial x} - \frac{\partial g_{12}}{\partial x}) + \frac{g^{12}}{2}(\frac{\partial g_{12}}{\partial y} + \frac{\partial g_{22}}{\partial x} - \frac{\partial g_{12}}{\partial y}) = \frac{y^2}{2}(-\frac{2}{y^3}) = -\frac{1}{y}
	\end{align*}

	Computation follows similarly for $\Gamma_{22}^2$ and $\Gamma_{11}^2$. 
	%?Should I work this out fully?
	
\end{proof}

Let $V_0 = (0,1)^T$ tangent vector at $(0,1) \in \R_+^2$. Let $V(t)$ be the parallel transport of $V_0$ along curve $x = t,y=1$. 

\begin{prop}
	V(t) makes angle t with direction of y-axis(measured clockwise)
\end{prop}

\begin{proof}
	Set $v(t) = (a(t),b(t))$

	We know
	\begin{align*}
		\frac{\partial a}{\partial t} + \Gamma_{12}^1 b = 0 \\
		\frac{\del b}{\del t} + \Gamma_{11}^2a = 0
	\end{align*}

	We have $a = cos \theta(t), b = sin \theta(t)$ and long the curve we have $y = 1$, we obtain from the equations above that $\frac{d \theta}{dt} = -1$. With $v(0) = v_0$ then $\theta(t) = \pi/2 - t$. 
\end{proof}

\question{Question 4}

Let $L : T_p(\m) \to T_p(\m)$ via $L(Y) = \nabla_Y X|_p$ which is linear. Then write $DIV(X) = trace(L)$ since $L$ is a linear operator on finite dimensional vector spaces.

\begin{prop}
	$DIV(X) = div(X) = \frac{d i_X (\omega_g)}{\omega_g}$
\end{prop}

\begin{proof}

	$DIV(X)$ is defined as the trace of L where $L$ takes $Y \to \nabla_Y X$. 

	We follow do Carmo's plan. 

	%?Do we have to show we can find a geodesic frame at every point?

	First we establish we can write at a point $p \in \m$
	\begin{align*}
		DIV X = \sum_i E_i(f_i)(p)
	\end{align*}

	where $X = f^i E_i$ with $\{E_i\}$ an orthonormal basis s.t. at $p$.

	Note for each $E_i$, $a_i^j E = \nabla_{E_i} X = \nabla_{E_i} f^i E_i = E^i[f] E_i  $
	\begin{align*}
		trace(L) = \sum_i a_i^i
	\end{align*}

	But note exactly $E_i(f_i)(p) = a_i^i$. 

	Now pick 1-forms $\omega_i$ s.t. $\omega = \omega_i(E_j) = \delta_{i,j}$. Since $\m$ is oriented this is a volume form via an argument similar to the direction orientability $\implies$ existance of volume form. In particular we know $dx_1 \wedge ... \wedge dx_n$ can be well defined globally as a volume form, which is equivalent to $f\omega_1 \wedge ... \wedge \omega_n$ for some $f$ which we can take to be positive by negating appropriate $E_i$ and $\omega_i$. 
	%?Should I be more careful here?

	Set $\theta_i = \omega_1 \wedge ... \wedge \hat{\omega_i} \wedge ... \wedge \omega_n$. We compute

	\begin{align*}
		i_X(\omega)(v_2,...,v_n) &= \omega(X,v_2,...,v_n) = \omega(f^iE_i,v_2,...,v_n) = \sum_i \omega(f_iE_i,v_2,...,v_n) \\
		&=\sum_i \omega(f_iE_i,v_2,...,v_n) = \sum_i (-1)^{i-1}\omega_i(f_iE_i) \theta_i(v_2,...,v_n)
	\end{align*}

	since if we have $\omega_j(f_iE_i)$ for some $j \neq i$ then we have 0 in a neighborhood. Note $\omega_i(f_i E_i) = f_i$ at a point. So

	\begin{align*}
		\sum_i (-1)^{i-1}\omega_i(f_iE_i) \theta_i(v_2,...,v_n) = \sum_i (-1)^{i+1}f_i \theta_i(v_2,...,v_n)
	\end{align*}

	which gives $i_X(\omega) = \sum_i (-1)^{i+1} f_i \theta_i$

	Then write
	\begin{align*}
		d(i(X)\omega) = \sum_i (-1)^{i+1} df_i \wedge \theta_i + \sum_i (-1)^{i+1} f_i \wedge d\theta_i = \sum_i E_i(f_i) \omega + \sum_i (-1)^{i+1}f_i \wedge d\theta_i
	\end{align*}
\end{proof}

Note $d \theta_i = 0$ since

\begin{align*}
	d\omega_k (E_i,E_j) = E_i \omega_k(E_j) - E_j \omega(E_i) - \omega_k([E_i,E_j]) = \omega_k (\nabla_{E_i} E_j - \nabla_{E_j} E_i)
\end{align*}

So really 

\begin{align*}
	d(i(X)\omega)(p) = \sum_i E_i(f_i)(p) \omega = div X(p) \omega
\end{align*}


\end{document}

