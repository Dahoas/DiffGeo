\documentclass[11pt]{article}
%you can look for fun LaTeX packages to use hereasdf

\usepackage{amsmath}
\usepackage{amssymb}
\usepackage{fancyhdr}
\usepackage{amsthm}

\usepackage{graphicx}
\usepackage{dcolumn}
\usepackage{bm}

%fun commands for fun sets
%make sure to use these in math mode
\newcommand{\Z}{\mathbb{Z}}
\newcommand{\R}{\mathbb{R}}
\newcommand{\N}{\mathbb{N}}
\newcommand{\C}{\mathbb{C}}
\newcommand{\m}{\mathcal{M}}
\newcommand{\Tt}{\mathcal{T}}
\newcommand{\pa}{\partial}
\newcommand{\Dd}{\mathcal{D}}
\newcommand{\del}{\partial}



\oddsidemargin0cm
\topmargin-2cm    
\textwidth16.5cm   
\textheight23.5cm  

\newcommand{\question}[2] {\vspace{.25in} \hrule\vspace{0.5em}
\noindent{\bf #1: #2} \vspace{0.5em}
\hrule \vspace{.10in}}
\renewcommand{\part}[1] {\vspace{.10in} {\bf (#1)}}

\newcommand{\myname}{Alex Havrilla}
\newcommand{\myandrew}{alumhavr}
\newcommand{\myhwnum}{Hw 7}

\newtheorem{prop}{Prop}

\setlength{\parindent}{0pt}
\setlength{\parskip}{5pt plus 1pt}
 
\pagestyle{fancyplain}
\lhead{\fancyplain{}{\textbf{HW\myhwnum}}}      % Note the different brackets!
\rhead{\fancyplain{}{\myname\\ \myandrew}}
\chead{\fancyplain{}{\mycourse}}

\linespread{1.3}

\begin{document}

\medskip                        

\thispagestyle{plain}
\begin{center}

{\myname}

\myandrew

\myhwnum

\end{center}

\question{Question 1}

\begin{prop}
	Let M be a riemanian manifold with sectional curvature identically zero. Then for every $p \in M$ the mapping
	\begin{align*}
		exp_p : B_{\epsilon}(0) \subseteq T_p M \to B_{\epsilon}(p)
	\end{align*}
	is an isometry where $B_{\epsilon}(p)$ is a normal ball. 
\end{prop}

\begin{proof}

	The idea is to establish an isometry by calculating solutions to a jacobi field satisfying $\frac{d^2 J}{dt^2}$ via 0 sectional curvature.	

	Let arbitrary $p \in M$. We show $exp_p$ is an isometry. This is clearly a diffeomorphism so suffices to show riemannian metric is preserved. 

	Consider $v \in B_{\epsilon}(0)$ with $\hat{v} \in T_vT_pM$. Set
	\begin{align*}
		\gamma_s(t) = exp_p(t(v+s\hat{v})) 
	\end{align*}
	Then $J(t) = \frac{d\gamma_s}{ds}|_{s=0}$ is a jacobi field s.t. $J(t) = (d exp_p)(t\hat{v})$
	%How do we even compute this?

	Because we have 0 sectional curvature we know a jacobi field satisfies $\frac{d^2J}{dt^2} = 0$ via the jacobi equation. 

	Set an ON frame $E_i$ an ON frame at $T_pM$ extended via parallel transport. Write $J(t) = a^i E_i$ and $\hat{v} = \hat{v}^i E_i$ where we identify tangent spaces of tangent spaces with the tangent space. Then since $\frac{d^2 J}{dt^2} = 0$ we know $J(t) = (\hat{v}^i t)E_i(t)$ because $\frac{d^2 J}{dt^2} = (a^i)''(t)E_i(t)$ with $\frac{DJ}{dt}(0) = \hat{v}$. 
	%How do we show this?
	Then $\hat{v}^iE_i(1) = (dexp_p)(\hat{v})$. This allows us to compute for additional $\hat{u} \in T_vT_pM$ 
	\begin{align*}
		\langle (d exp_p)(\hat{u}),(d exp_p (\hat{v})) \rangle = \hat{u}^i \hat{v}^j \langle E_i(1),E_j(1)\rangle = \langle \hat{u},\hat{v}\rangle
	\end{align*}

	which establishes an isometry.
\end{proof} 

\question{Question 2}

\begin{prop}
	Let M be a Riemannian manifold, $\gamma : [0,1] \to M$ a geodesic, and J a jacobi field along $\gamma$. Then there exists a parameterized surface $f(t,s)$ where $f(t,0) = \gamma(t)$ and curves $t \to f(t,s)$ are geodesics s.t. $J(t) = \frac{df}{ds}(t,0)$
\end{prop}

\begin{proof}
	%Ideas: Define f by radiating geodesics outward exp_\lambda(s) tW(s) along vertical line

	The idea is to construct f by flowing geodesics out along a "vertical" strip.

	Choose curve $\lambda(s)$ in M s.t. $\lambda(0) = \gamma(0)$ with $\lambda'(0) = J(0)$. Along $\lambda$ choose vector field $W(s)$ with $W(0) = \gamma'(0)$, $\frac{DW}{ds}(0) = \frac{DJ}{dt}(0)$. We set $f(s,t) = exp_{\lambda(s)} tW(s)$. This defines a parameterized surface. We show the curves are geodesics on it. 

	We know $\frac{d\lambda}{ds} = \lambda'(0) = J(0)$. Further we also have $\frac{df}{ds}(0,0) = \lambda'(0)$ since $\frac{df}{ds}(0,0) = \frac{d}{ds}|_{s=0}exp_{\lambda(s)}(0) = \frac{d}{ds}|_{s=0} \lambda(s) = \lambda'(s)$. 
	%Using principle of subsitution to simplify things

	We also show
	\begin{align*}
		\frac{D}{dt}\frac{df}{ds} (0,0) = \frac{D}{ds}\frac{df}{dt}(0,0) = \frac{DW}{ds}(0) = \frac{DJ}{dt}(0)
	\end{align*}

	We know by construction the third equality holds. The first equality holds by the symmetry lemma. Then we show the second holds.

	Compute 
	%Don't really know why these derivative relations hold but oh well?
	\begin{align*}
		\frac{D}{ds}\frac{df}{dt}(0,0) = \frac{D}{ds}|_{s=0}[\frac{d}{dt}|_{t=0} exp_{\lambda(0)}tW(s)] = \frac{D}{ds}|_{s=0}[W(s)] = \frac{DW}{ds}(0)
	\end{align*}
	%This probably isn't very rigorous

	This establishes the interesting curves and further $J(t) = \frac{df}{ds}(t,0)$ via the initial conditions we've shown. Thus we finish the proof.
\end{proof}
	
\question{Question 3}

\begin{prop}
	Let M be a Riemannian manifold with non-postive sectional curvature. Then for all p, the conjugate cut locus $C(p)$ is empty.
\end{prop}

\begin{proof}
	Fix geodesic $\gamma : [0,a] \to M$ with $\gamma(0) = p \in M$. Suppose we have non-trivial $J$ jacobi field on $\gamma$ with $J(0) = J(a) = 0$. We have the jacobi equation:
	\begin{align*}
		0 = \frac{D^2 J}{dt^2} + R(\gamma',J)\gamma'
	\end{align*}

	which tells us 
	\begin{align*}
		\frac{d}{dt}\langle \frac{DJ}{dt},J\rangle = \langle \frac{D^2J}{dt^2},J\rangle + \langle \frac{DJ}{dt},\frac{DJ}{dt}\rangle \geq 0
	\end{align*}

	since cleary the second term is nonnegative and the first term is equal to the negation of sectional curvature via the jacobi equation, where we know sectional curvature nonpositive.

	But then since $J(0) = J(a) = 0$ it must be $\langle DJ dt,J\rangle = 0$. Then with $\frac{d}{dt} \langle J,J\rangle = 2 \langle DJ/dt,J\rangle = 0$ it must be J is 0.
\end{proof}

\question{Question 4}

Let M be a riemannian manifold of dimension two(so M is a surface). Let $B_{\delta}(p)$ be a normal ball around the point $p \in M$. Consider the parameterized surface 
	\begin{align*}
		f(\rho,\theta) = exp_p (\rho v(\theta))
	\end{align*}
	where $v(\theta)$ is a circle of radius $\delta$ in $T_p M$ parameterized by the central angle $\theta$

\begin{prop}
	$(\rho,\theta)$ are coordinates in an open set $U$ formed by open ball $B_{\delta}(p)$ minus ray $exp_p(-\rho v(0))$ with $0 < \rho < \delta$. These are polar coordinates
\end{prop}

\begin{proof}
	Inuitively $v(\theta)$ gives the direction we flow and $\rho$ gives the distance we flow(from p). 

	We know $exp_p$ diffeomorphic from ball in $T_pM$ around 0 to $B_{\delta}(p)$. The uniqueness of coordinates is given by the uniqueness of geodesics. 

	The ray $exp_p(-\rho v(0))$ for $0 < \rho < \delta$ is not included because $-\pi,\pi$ are not valued.
\end{proof}

\begin{prop}
	$g_{ij}$ of the riemannian metric are $g_{12} = 0$, $g_{11} = |\frac{df}{d\rho}|^2 = |v(\theta)|^2 = 1$ and $g_{22} = |\frac{df}{d\theta}|^2$
\end{prop}

\begin{proof}
	Recall $g_{ij} = \langle \frac{d}{dx_i},\frac{d}{dx_j}\rangle$. Gauss's lemma tells us 
	\begin{align*}
		\langle (dexp_p)_v(v),(d exp_p)_v(w) \rangle  = \langle v,w\rangle
	\end{align*}

	Use this to compute
	%All info about coordinates taken care of by considering d/d \rho, d/d \theta
	\begin{align*}
		g_{11} = \langle \frac{d}{d \rho}, \frac{d}{d \rho}\rangle = \langle \frac{d f}{d \rho}, \frac{d f}{d \rho}\rangle = |\frac{d f}{d \rho}|^2
	\end{align*}

	by definition. Similarly $g_{22} = |\frac{d f}{d\theta}|^2$.

	\begin{align*}
		g_{12} = \langle \frac{d}{d \rho}, \frac{d}{d \theta} \rangle = 0 
	\end{align*}

	similarly and finally%why?

	\begin{align*}
		|\frac{df}{d\rho}|^2 = |\frac{d}{dt}|_{t=0}(exp_p)((\rho + t)v(\theta))|^2 = |v(\theta)|^2 = 1
	\end{align*}
	%Why?

\end{proof}

\begin{prop}
	Along the geodesic $f(\rho,0)$ we have
	$(\sqrt{g_{22}})_{\rho \rho} = -K(p)\rho + R(\rho)$ where $lim_{\rho \to 0} \frac{R(\rho)}{\rho} = 0$ and $K(p)$ is the sectional curvature of M at p
\end{prop}

\begin{proof}
	Let $f(\rho,0)$ be $\gamma(\rho)$. Note $\frac{df}{d\theta}$ is a jacobi field. %why? - has functional form 
	$g_{22}=|\frac{df}{d\theta}|^2$ can be taylor expanded as $\rho^2 - (1/3)K(\rho,\sigma)\rho^4 + ... G(\rho)$. But we need $\sqrt{g_{22}}$ so we taylor expand to get $\sqrt{g_{22}} = \rho - (1/6)K(\rho,\sigma)\rho^3 + ... R(\rho)$ which we get by noting $\sqrt{g_{22}} * \sqrt{g_{22}} = g_{22}$ and collecting terms. Differentiating and noting $R''(\rho)$ is small gives the desired result. 

	Writing $\sqrt{g_{22}}_{\rho \rho}s$ out for the next problem:
	\begin{align*}
		\sqrt{g_{22}}_{\rho \rho} = -K(\rho,\sigma)\rho + R''(\rho)
	\end{align*}
\end{proof}

\begin{prop}
	$lim_{\rho \to 0} \frac{\sqrt{g_{22}}_{\rho \rho}}{\sqrt{g_{22}}} = - K(p)$
\end{prop}

\begin{proof}
	This follows directly from the above proposition and a taylor expansion for $\sqrt{g}_{22}$
\end{proof}

%This shows gaussian curvtaure is same as sectional curvature in two dimensions
	
\end{document}

